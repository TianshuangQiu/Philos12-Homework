\documentclass[12pt]{article}
\usepackage[usenames]{color} %used for font color
\usepackage{amsmath, amssymb, amsthm}
\usepackage{wasysym}
\usepackage[utf8]{inputenc} %useful to type directly diacritic characters
\usepackage{graphicx}
\usepackage{caption}
\usepackage{subcaption}
\usepackage{float}
\usepackage{mathtools}
\usepackage [english]{babel}
\usepackage [autostyle, english = american]{csquotes}
\MakeOuterQuote{"}
\graphicspath{ {./} }
\newcommand{\Z}{\mathbb{Z}}
\newcommand{\N}{\mathbb{N}}
\newcommand{\R}{\mathbb{R}}
\newcommand{\Q}{\mathbb{Q}}
\newcommand{\prob}{\mathbb{P}}
\newcommand{\degrees}{^{\circ}}
\DeclarePairedDelimiter\ceil{\lceil}{\rceil}
\DeclarePairedDelimiter\floor{\lfloor}{\rfloor}

\author{Tianshuang (Ethan) Qiu}
\begin{document}
\title{Philosophy 12, Problem Set 1}
\maketitle
\newpage

\section{What is Prop. Logic?}
\subsection{Q1.0}
\begin{itemize}
    \item If you are taking this class, then you are a Berkeley student.
    \item It is not the case that you are not taking this class.
    \item Therefore it is not the case that you are a Berkeley student.
\end{itemize}
This is not a good argument since there are a lot of Berkeley students who 
are not in this class.

\subsection{Q1.1}
\begin{itemize}
    \item It is not the case that you can be both at least 10 years old and less than 10 years old.
    \item You are at lest 10 years old.
    \item Therefore you are not less than 10 years old.
\end{itemize}
This is a good argument, the conclusion logically follows from the premise.

\section{Truth-Functional Connectives}
\subsection{Q2}

\begin{tabular}{ | c | c | c |}
    \hline
    $p$ & Pete heard that $p$ \\
    \hline
    TRUE & Maybe (didn't hear it) \\
    \hline
    FALSE & Maybe (false rumors) \\
    \hline
\end{tabular}
\newline
As we can see, the truth table cannot be written because the truth value of 
"Pete heard that $p$" does not only depend on $p$
    
\subsection{Q3}

\begin{tabular} {|c|c|c|}
    \hline
    Kate in library & Kate in gym & Kate in neither library nor gym \\
    \hline
    TRUE & TRUE & FALSE \\
    \hline
    TRUE & FALSE & FALSE \\
    \hline
    FALSE & TRUE & FALSE \\
    \hline
    FALSE & FALSE & TRUE \\
    \hline
\end{tabular}
\newline
The truth table is written above. The connective's value depends only on the operand's. 
Therefore it is truth functional.

\section{The Truth-Functional Conditional}
\subsection{Q4}
\textbf{Statement 1}: If I am not playing tennis, then I am watching tennis.
\newline
\textbf{Statement 2}: If it is not the case that I am watching tennis, then I am reading about tennis.
Since we can only do one activity at a time, we have the following truth table.
\newline
\begin{tabular}{|c|c|c|}
    \hline 
    Playing Tennis & Watching Tennis & Reading about Tennis \\
    \hline
    TRUE & FALSE & FALSE \\
    \hline
    FALSE & TRUE & FALSE \\
    \hline
    FALSE & FALSE & TRUE \\
    \hline
\end{tabular}
\newline
The first line is invalid since he is not watching tennis AND not reading about tennis, 
contradicting Statement 2.
\newline
The second line is valid since it satisfies both statements.
\newline
The thrid line is invalid since he is not watching tennis AND not watching tennis, contradicting 
Statement 1.

\subsection{Q5}
\begin{tabular}{|c|c|c|}
    \hline 
    Person is actually a knight & Statement is True  & Is there Gold \\
    \hline
    TRUE & TRUE & TRUE \\
    \hline
    FALSE & FALSE & ? \\
    \hline
\end{tabular}
\newline
The trivial part is if we have already met a knight, then he tells us 
the truth and there is gold on the island. 
\newline
The knave's case is more interesting, he lies and this statement is false. Therefore the predicate must be true 
and the conclusion false. However the only way for the predicate to be true is 
if he is a knight, which is a contradiction. Therefore this case is not possible.
\newline
Since the knave's case is not logically possible, we must have met a knight and there is gold on the island.

\section{Valid Forms of Argument I}
\subsection{Q6.1}
Let $p$ be "rabbits are fish."
\newline
Let $q$ be "mice are fish."
\newline
Let $r$ be "unicors exist."

\begin{itemize}
    \item $p \implies q$
    \item $q \implies r$
    \item $p \implies r$
\end{itemize}
The argument is valid but unsound. When $p$ is true, $q$ is also true, 
and thus $r$ is always true. However it is not sound since $p$ is always false.

\subsection{Q6.2}
Let $p$ be "rabbits are fish."
\newline
Let $q$ be "mice are fish."
\newline
Let $r$ be "unicorns exist."

\begin{itemize}
    \item $p \lor q$
    \item $p \lor r$
    \item $p \land (q \lor r)$
\end{itemize}
We can distribute the and operator to get $(p \land q) \lor (p \land r)$, 
which is not equal to both our premises ($(p \lor q) \land (p \lor r)$).
Therefore it invalid and unsound.

\section{Valid Forms of Argument II}
\subsection{Q7.1}

\begin{tabular}{|c|c|c|c|c|c|}
    \hline 
    $q$ & $\neg q$ & $q \lor \neg q$ & $r$ & $p$ & $\neg r \implies \neg p$ \\
    \hline
    FALSE & TRUE & TRUE & TRUE & TRUE & $\text{FALSE} \implies \text{FALSE}$ \\
    \hline
\end{tabular}
\newline
$\text{FALSE} \implies \text{FALSE}$ is true since false implying anything is true. Thus here 
we have all the premises true, but the conclusion $q$ is false. Thus 
the argument is invalid.

\subsection{Q7.2}
\begin{tabular}{|c|c|c|c|c|c|}
    \hline
    $p$ & $q$ & $\neg p$ & $p \land \neg p$ & $\text{Statement 1} \implies q$\\
    \hline
    TRUE & TRUE & FALSE & FALSE & TRUE \\
    \hline
    TRUE & FALSE & FALSE & FALSE & TRUE \\
    \hline
    FALSE & TRUE & TRUE & FALSE & TRUE \\
    \hline
    FALSE & FALSE & TRUE & FALSE & TRUE \\
    \hline
\end{tabular}
\newline
Every row evavluates to true, and the conclusion is true.


\end{document}