\documentclass[12pt]{article}
\usepackage[usenames]{color} %used for font color
\usepackage{amsmath, amssymb, amsthm}
\usepackage{wasysym}
\usepackage[utf8]{inputenc} %useful to type directly diacritic characters
\usepackage{graphicx}
\usepackage{caption}
\usepackage{subcaption}
\usepackage{float}
\usepackage{mathtools}
\usepackage [english]{babel}
\usepackage [autostyle, english = american]{csquotes}
\MakeOuterQuote{"}
\graphicspath{ {./} }
\newcommand{\Z}{\mathbb{Z}}
\newcommand{\N}{\mathbb{N}}
\newcommand{\R}{\mathbb{R}}
\newcommand{\Q}{\mathbb{Q}}
\newcommand{\prob}{\mathbb{P}}
\newcommand{\degrees}{^{\circ}}
\DeclarePairedDelimiter\ceil{\lceil}{\rceil}
\DeclarePairedDelimiter\floor{\lfloor}{\rfloor}

\author{Tianshuang (Ethan) Qiu}
\begin{document}
\title{Philosophy 12, Problem Set 1}
\maketitle

\section{Q1}
\subsection{1}
No it is not, when $\phi = 0, \psi = 1$, $\phi \implies \psi$ is ture,
but $\psi \implies \phi$ is false. Therefore they are not equivalent.

\subsection{2}
\begin{tabular}{ | c | c | c | c|}
    \hline
    $\phi$ & $\psi$ & $\phi \implies \psi$ & $\neg\psi \implies \neg\phi$\\
    \hline
    1 & 1 & 1 & 1 \\
    \hline
    1 & 0 & 0 & 0 \\
    \hline
    0 & 0 & 1 & 1 \\
    \hline
    0 & 1 & 1 & 1 \\
    \hline
\end{tabular}
\newline
Since every line is the same, they are equivalent.

\subsection{3}
Let $\phi = 0, \psi = 1$, $(\phi \implies \psi) = 0$ so $\neg (\phi \implies \psi) = 1$.
However $\phi \lor \neg \psi = 0$. They are not equivalent.

\subsection{4}
\begin{tabular}{ | c | c | c | c|}
    \hline
    $\phi$ & $\psi$ & $\neg(\phi \implies \psi)$ & $\phi \land \neg \psi$\\
    \hline
    1 & 1 & 0 & 0 \\
    \hline
    1 & 0 & 1 & 1 \\
    \hline
    0 & 0 & 0 & 0 \\
    \hline
    0 & 1 & 0 & 0 \\
    \hline
\end{tabular}
\newline
Since every line is the same, they are equivalent.

\subsection{5}
Let $\phi = 0, \psi = 1$, $(\phi \iff \psi) = 0$ so $\neg(\phi \iff \psi) = 1$.
However $\neg\phi \iff \neg \psi = 0$. They are not equivalent.

\subsection{6}
\begin{tabular}{ | c | c | c | c|}
    \hline
    $\phi$ & $\psi$ & $\neg(\phi \iff \psi)$ & $\neg \phi \iff \psi$\\
    \hline
    1 & 1 & 0 & 0 \\
    \hline
    1 & 0 & 1 & 1 \\
    \hline
    0 & 0 & 0 & 0 \\
    \hline
    0 & 1 & 1 & 1 \\
    \hline
\end{tabular}
\newline
Since every line is the same, they are equivalent.

\subsection{7}
\begin{tabular}{ | c | c | c | c|}
    \hline
    $\phi$ & $\psi$ & $(\phi \land \psi) \iff (\phi \lor \psi)$ & $\phi \iff \psi$\\
    \hline
    1 & 1 & 0 & 0 \\
    \hline
    1 & 0 & 0 & 0 \\
    \hline
    0 & 0 & 1 & 1 \\
    \hline
    0 & 1 & 0 & 0 \\
    \hline
\end{tabular}
\newline
Since every line is the same, they are equivalent.


\section{Q2}
\subsection{1}
\begin{tabular}{ | c | c | c | c |}
    \hline
    $q$ & $r$ & $\neg(q \land r)$ & $\neg r$ \\
    \hline
    1 & 1 & 0 & 0 \\
    \hline
    1 & 0 & 1 & 1 \\
    \hline
    0 & 0 & 1 & 1 \\
    \hline
    0 & 1 & 1 & 0 \\
    \hline
\end{tabular}
\newline
In all possible rows, there is no such row where both premises are true and the
conclusion false, therefore this is a valid consequence.

\subsection{2}
\begin{tabular}{ | c | c | c | c | c |}
    \hline
    $p$ & $q$ & $r$ & $\neg p \lor \neg q \lor \neg r$ & $q \lor r$ \\
    \hline
    1 & 1 & 1 & 1 & 1 \\
    \hline
    1 & 1 & 0 & 0 & 1 \\
    \hline
    1 & 0 & 1 & 1 & 1 \\
    \hline
    1 & 0 & 0 & 1 & 0 \\
    \hline
    0 & 1 & 1 & 1 & 1 \\
    \hline
    0 & 1 & 0 & 1 & 1 \\
    \hline
    0 & 0 & 1 & 1 & 1 \\
    \hline
    0 & 0 & 0 & 1 & 0 \\
    \hline
\end{tabular}
\newline
In all possible rows, there is no such row where all premises are true and the
conclusion false, therefore this is a valid consequence.


\section{Q3}
\subsection{a}
Let "vinegar is included in the batter" be $p$, "baking soda is included in the batter" be $q$,
"the velvet cake rises" be $r$.
\newline
Then we have $(p \land q) \implies r$. Our conclusion is that $\neg r \implies (\neg p \implies q)$.
\newline
This statement is valid because either the vinegar or the baking soda must be absent. If the batter contained
vinegar, then it must not have baking soda.

\subsection{b}
Let "Kovak wins the election" be $p$, "the taxes increase" be $q$,
"her party maintains control of the legislature" be $r$.
\newline
Then we have $p \implies (r \implies q)$. Our conclusion is that $\neg q \implies (\neg p \land \neg r)$.
\newline
This statement is not valid because Kovak could have won the election but her party did not maintain control. This
satisfies the premise but contradicts the conclusion.

\subsection{c}
Let $a = 0$ be $p$, $b = 0$ be $q$, $a+b =0$ be $r$.
\newline
Then we have $(p \land q) \implies r$. Our conclusion is $\neg r \implies (\neg q \lor \neg p)$.
\newline
This statement is valid because it is the contrapositive of the original.

\subsection{d}
Let "Jones drove the car" be $p$, "Smith is innocent" be $q$,
"Brown fired the gun" be $r$.
\newline
Then we have $(p \implies q) \land (\neg r \implies \neg q)$. Our conclusion is $r \implies \neg q$.
\newline
This statement is invalid because in the case that Brown fired the gun, Smith is innocent, and Jones drove the car,
all premises are met. However the conclusion is not: Jones did drive the car. Therefore the conclusion
is invalid.


\section{Q4}
\subsection{a}
\begin{tabular}{ | c | c | c | c | c | c | c | c | c |}
    \hline
    $p$ & $q$ & $r$ & $\neg p$ & $q \implies p$ & $p \implies r$ & $q \implies r $ & $\neg q$ & $\neg r$\\
    \hline
    0 & 0 & 0 & 1 & 1 & 1 & 1 & 1 & 1\\
    \hline
    0 & 0 & 1 & 1 & 1 & 1 & 1 & 1 & 0\\
    \hline
    0 & 1 & 0 & 0 & 0 & 1 & 1 & 0 & 1\\
    \hline
    0 & 1 & 1 & 1 & 0 & 1 & 1 & 0 & 0\\
    \hline
    1 & 0 & 0 & 0 & 1 & 0 & 1 & 1 & 0\\
    \hline
    1 & 0 & 1 & 0 & 1 & 1 & 1 & 1 & 1\\
    \hline
    1 & 1 & 0 & 0 & 1 & 0 & 0 & 0 & 1\\
    \hline
    1 & 1 & 1 & 0 & 1 & 1 & 1 & 0 & 0\\
    \hline
\end{tabular}
\newline
We can see that the first two lines are the only cases where all 3 premises are true.
In these cases both (i) and (ii) are true but (iii) is false on line 2. Therefore (i)
and (ii) are logically implied.

\subsection{b}
\begin{tabular}{ | c | c | c | c | c | c | c | c | c |}
    \hline
    $p$ & $q$ & $r$ & $p \lor r$ & $q \implies \neg r$ & $q \lor \neg r$ & $p \implies q$ & $p \implies (r \lor \neg q)$ & $(\neg p \lor r) \implies q$\\
    \hline
    0 & 0 & 0 & 0 & 1 & 1 & 1 & 1 & 0\\
    \hline
    0 & 0 & 1 & 1 & 1 & 0 & 1 & 1 & 0\\
    \hline
    0 & 1 & 0 & 0 & 1 & 0 & 1 & 1 & 1\\
    \hline
    0 & 1 & 1 & 1 & 0 & 1 & 1 & 1 & 1\\
    \hline
    1 & 0 & 0 & 1 & 1 & 1 & 0 & 1 & 1\\
    \hline
    1 & 0 & 1 & 1 & 1 & 0 & 0 & 1 & 0\\
    \hline
    1 & 1 & 0 & 1 & 1 & 1 & 1 & 0 & 1\\
    \hline
    1 & 1 & 1 & 1 & 0 & 1 & 1 & 1 & 1\\
    \hline
\end{tabular}
\newline
We can see that there are only 2 rows in which all 3 prermises are satisfied. In these rows we observe that
only (iii) is always true. Therefore only (iii) is logically implied.


\section{Q5}
\subsection{a}
No
\subsection{b}
No
\subsection{c}
Yes
\subsection{d}
Yes
\subsection{e}
No
\subsection{f}
Yes
\subsection{g}
No
\subsection{h}
No
\subsection{i}
Yes
\subsection{j}
Yes
\subsection{k}
Yes
\subsection{l}
No

\section{Q6}
\subsection{Base case}
Consider the proposition $\phi$. It has length $1$ and no brackets, therefore we know that
$p_\phi = 0, l_\phi = 1$. The base case holds.

\subsection{Inductive case}
Assume that $2p_\psi<l_\psi$ holds for all $\psi \in L$ such that $l_\psi \leq n$ for some natural $n$.
\newline
Let $A, B$ be arbitrary formulas of length $n$. By our hypothesis we know that they satisfy $2p_A<l_A$ (same for $B$). Furthermore, we know that formulas can be formed by joining two sub-formulas with $(A \land B)$, $(A \lor B)$,
$(A \implies B)$ or $\neg A$. In the first three cases, our total amount of brackets is $p_A+p_B+2$ and our length is $l_A+l_B+3$. Now we can multiply the first statement by 2: $2p_A+2p_B+4$. Now we can observe the second statement. By our inductive hypothesis we know that $l_A+l_B+3 > 2p_A+2p_B+3$. However, since there can only be an integer number of characters, we have $l_A \geq 2p_A + 1$. The same is true for $l_B$. Therefore we have $l_A+l_B+3 \geq 2p_A+2p_B+3+2 > 2p_A+2p_B+3 $
In the last case, we have a total of $p_A$ brackets and a total lengtht of $l_A+1$. It is obviously true due to our inductive hypothesis.
\newline
Thus we have proven the base case and the inductive case.
\newline
Q.E.D.

\end{document}
