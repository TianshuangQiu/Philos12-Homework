\documentclass[12pt]{article}
\usepackage[usenames]{color} %used for font color
\usepackage{amsmath, amssymb, amsthm}
\usepackage{wasysym}
\usepackage[utf8]{inputenc} %useful to type directly diacritic characters
\usepackage{graphicx}
\usepackage{caption}
\usepackage{subcaption}
\usepackage{float}
\usepackage{mathtools}
\usepackage [english]{babel}
\usepackage [autostyle, english = american]{csquotes}
\MakeOuterQuote{"}
\graphicspath{ {./} }
\newcommand{\Z}{\mathbb{Z}}
\newcommand{\N}{\mathbb{N}}
\newcommand{\R}{\mathbb{R}}
\newcommand{\Q}{\mathbb{Q}}
\newcommand{\prob}{\mathbb{P}}
\newcommand{\degrees}{^{\circ}}
\DeclarePairedDelimiter\ceil{\lceil}{\rceil}
\DeclarePairedDelimiter\floor{\lfloor}{\rfloor}

\author{Tianshuang (Ethan) Qiu}
\begin{document}
\title{Philosophy 12, Problem Set 1}
\maketitle

\section{Q1}
\subsection{1}
No it is not, when $\phi = 0, \psi = 1$, $\phi \implies \psi$ is ture, 
but $\psi \implies \phi$ is false. Therefore they are not equivalent.

\subsection{2}
\begin{tabular}{ | c | c | c | c|}
    \hline
    $\phi$ & $\psi$ & $\phi \implies \psi$ & $\neg\psi \implies \neg\phi$\\
    \hline
    1 & 1 & 1 & 1 \\
    \hline
    1 & 0 & 0 & 0 \\
    \hline
    0 & 0 & 1 & 1 \\
    \hline
    0 & 1 & 1 & 1 \\
    \hline
\end{tabular}
\newline
Since every line is the same, they are equivalent.

\subsection{3}
Let $\phi = 0, \psi = 1$, $(\phi \implies \psi) = 0$ so $\neg (\phi \implies \psi) = 1$. 
However $\phi \lor \neg \psi = 0$. They are not equivalent.

\subsection{4}
\begin{tabular}{ | c | c | c | c|}
    \hline
    $\phi$ & $\psi$ & $\neg(\phi \implies \psi)$ & $\phi \land \neg \psi$\\
    \hline
    1 & 1 & 0 & 0 \\
    \hline
    1 & 0 & 1 & 1 \\
    \hline
    0 & 0 & 0 & 0 \\
    \hline
    0 & 1 & 0 & 0 \\
    \hline
\end{tabular}
\newline
Since every line is the same, they are equivalent.

\subsection{5}
Let $\phi = 0, \psi = 1$, $(\phi \iff \psi) = 0$ so $\neg(\phi \iff \psi) = 1$. 
However $\neg\phi \iff \neg \psi = 0$. They are not equivalent.

\subsection{6}
\begin{tabular}{ | c | c | c | c|}
    \hline
    $\phi$ & $\psi$ & $\neg(\phi \iff \psi)$ & $\neg \phi \iff \psi$\\
    \hline
    1 & 1 & 0 & 0 \\
    \hline
    1 & 0 & 1 & 1 \\
    \hline
    0 & 0 & 0 & 0 \\
    \hline
    0 & 1 & 1 & 1 \\
    \hline
\end{tabular}
\newline
Since every line is the same, they are equivalent.

\subsection{7}
\begin{tabular}{ | c | c | c | c|}
    \hline
    $\phi$ & $\psi$ & $(\phi \land \psi) \iff (\phi \lor \psi)$ & $\phi \iff \psi$\\
    \hline
    1 & 1 & 0 & 0 \\
    \hline
    1 & 0 & 0 & 0 \\
    \hline
    0 & 0 & 1 & 1 \\
    \hline
    0 & 1 & 0 & 0 \\
    \hline
\end{tabular}
\newline
Since every line is the same, they are equivalent.


\section{Q2}
\subsection{1}
\begin{tabular}{ | c | c | c | c |}
    \hline
    $q$ & $r$ & $\neg(q \land r)$ & $\neg r$ \\
    \hline
    1 & 1 & 0 & 0 \\
    \hline
    1 & 0 & 1 & 1 \\
    \hline
    0 & 0 & 1 & 1 \\
    \hline
    0 & 1 & 1 & 0 \\
    \hline
\end{tabular}
\newline
In all possible rows, there is no such row where both premises are true and the 
conclusion false, therefore this is a valid consequence.

\subsection{2}
\begin{tabular}{ | c | c | c | c | c |}
    \hline
    $p$ & $q$ & $r$ & $\neg p \lor \neg q \lor \neg r$ & $q \lor r$ \\
    \hline
    1 & 1 & 1 & 1 & 1 \\
    \hline
    1 & 1 & 0 & 0 & 1 \\
    \hline
    1 & 0 & 1 & 1 & 1 \\
    \hline
    1 & 0 & 0 & 1 & 0 \\
    \hline
    0 & 1 & 1 & 1 & 1 \\
    \hline
    0 & 1 & 0 & 1 & 1 \\
    \hline
    0 & 0 & 1 & 1 & 1 \\
    \hline
    0 & 0 & 0 & 1 & 0 \\
    \hline
\end{tabular}
\newline
In all possible rows, there is no such row where all premises are true and the 
conclusion false, therefore this is a valid consequence.


\section{Q3}
\subsection{a}
Let "vinegar is included in the batter" be $p$, "baking soda is included in the batter" be $q$, 
"the velvet cake rises" be $r$. 
\newline
Then we have $(p \land q) \implies r$. Our conclusion is that $\neg r \implies (\neg p \implies q)$.
\newline
This statement is valid because either the vinegar or the baking soda must be absent. If the batter contained 
vinegar, then it must not have baking soda.

\subsection{b}
Let "Kovak wins the election" be $p$, "the taxes increase" be $q$, 
"her party maintains control of the legislature" be $r$. 
\newline
Then we have $p \implies (r \implies q)$. Our conclusion is that $\neg q \implies (\neg p \land \neg r)$.
\newline
This statement is not valid because Kovak could have won the election but her party did not maintain control. This 
satisfies the premise but contradicts the conclusion.

\subsection{c}
Let $a = 0$ be $p$, $b = 0$ be $q$, $a+b =0$ be $r$.
\newline
Then we have $(p \land q) \implies r$. Our conclusion is $\neg r \implies (\neg q \lor \neg p)$.
\newline
This statement is valid because it is the contrapositive of the original.

\subsection{d}
Let "Jones drove the car" be $p$, "Smith is innocent" be $q$, 
"Brown fired the gun" be $r$.
\newline
Then we have $(p \implies q) \land (\neg r \implies \neg q)$. Our conclusion is $r \implies \neg q$.
\newline
This statement is invalid because in the case that Brown fired the gun, Smith is innocent, and Jones drove the car, 
all premises are met. However the conclusion is not: Jones did drive the car. Therefore the conclusion 
is invalid.


\section{Q4}
\subsection{a}
For (i):
\newline
\begin{tabular}{ | c | c | c | c |}
    \hline
    $q$ & $r$ & $\neg(q \land r)$ & $\neg r$ \\
    \hline
    1 & 1 & 0 & 0 \\
    \hline
    1 & 0 & 1 & 1 \\
    \hline
    0 & 0 & 1 & 1 \\
    \hline
    0 & 1 & 1 & 0 \\
    \hline
\end{tabular}
\newline
\end{document}