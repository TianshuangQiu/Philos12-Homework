\documentclass[12pt]{article}
\usepackage[usenames]{color} %used for font color
\usepackage{amsmath, amssymb, amsthm}
\usepackage{wasysym}
\usepackage[utf8]{inputenc} %useful to type directly diacritic characters
\usepackage{graphicx}
\usepackage{caption}
\usepackage{subcaption}
\usepackage{float}
\usepackage{mathtools}
\usepackage [english]{babel}
\usepackage [autostyle, english = american]{csquotes}
\MakeOuterQuote{"}
\graphicspath{ {./} }
\newcommand{\Z}{\mathbb{Z}}
\newcommand{\N}{\mathbb{N}}
\newcommand{\R}{\mathbb{R}}
\newcommand{\Q}{\mathbb{Q}}
\newcommand{\prob}{\mathbb{P}}
\newcommand{\degrees}{^{\circ}}
\DeclarePairedDelimiter\ceil{\lceil}{\rceil}
\DeclarePairedDelimiter\floor{\lfloor}{\rfloor}

\author{Tianshuang (Ethan) Qiu}
\begin{document}
\title{Philosophy 12, Problem Set 1}
\maketitle

\section{Q1}

\subsection{1}
From the problem we have the following

\begin{equation}
    A \implies \neg B \land \neg L
\end{equation}
\begin{equation}
    B \iff D
\end{equation}
\begin{equation}
    \neg J \implies \neg C
\end{equation}
\begin{equation}
    D \implies K
\end{equation}
\begin{equation}
    L \implies M
\end{equation}
\begin{equation}
    D \implies \neg F
\end{equation}

$A, B, C, F$ has both $A$ and $B$, which contrardicts rule 1.
\newline
$A, C, D, E$ is a possible display of hats.
\newline
$A, D, E, F$ contradicts rule 2.
\newline
$B, C, D, F$ is a possible display of hats.
\newline
$B, C, E, F$ is a possible display of hats.

\subsection{2}
Following rule 6, $D$ must not be displayed. Therefore $B$ is not displayed. Therefore choice $B$ must be true.
$A, K, L, M$ are not certain.


\section{Q2}
\begin{tabular}{|c | c |c |c |c |c |}
    \hline
    $p$ & $q$ & $r$ & $(\neg p \implies q) \lor r$ & $\neg (p \implies q) \lor r$ & $\neg p \implies (q \lor r)$ \\
    \hline
    0 & 0 & 0 & 0 & 0 & 0 \\
    \hline
    0 & 0 & 1 & 1 & 1 & 1 \\
    \hline
    0 & 1 & 0 & 1 & 0 & 1 \\
    \hline
    0 & 1 & 1 & 1 & 1 & 1 \\
    \hline
    1 & 0 & 0 & 1 & 1 & 1 \\
    \hline
    1 & 0 & 1 & 1 & 1 & 0 \\
    \hline
\end{tabular}
\newline
We have found rows where 1,2 are different and 2, 3 are different. Thus these formulas are not equivalent.

\section{Q3}
Since $\phi$ is satisfiable, there exists a valuation $V$ such that $\hat{V}(\phi)=1$.
Then since $\psi$ is a logical consequence of $\phi$, we can use the same valuation $V$, and $\hat{V}(\psi)=1$. Thus $\psi$ is also satisfiable.

\section{Q4}
Base case: let $p$ be a proposition. Since $p^* = \neg p$ by definition, the base case holds.
\newline
Assume that for some statement $\phi, \psi$, we have $(\phi)^* = \neg \phi$ and $(\psi)^* = \neg \psi$.
\subsection{b}
Then consider $(\phi \land \psi)^*$. By definition we have $(\phi \land \psi)^*=(\phi^* \lor \psi^*)$.
By the inductive hypothesis it is equal to $(\neg \phi \lor \neg \psi)$. Thus we have proven the statement.

\subsection{c}
Consider $(\phi \lor \psi)^*$. By definition we have $(\phi \lor \psi)^*=(\phi^* \land \psi^*)$.
By the inductive hypothesis it is equal to $(\neg \phi \land \neg \psi)$. Thus we have proven the statement.


\section{Q5}
\subsection{a}
Assume that this statement is not a tautology, then $\hat{V}(p \implies q)=0$ and $\hat{V}(q \implies p) = 0$.
However we4 know that $p \implies q$ is false if and only if $p = 1, q = 0$, but this makes the latter true. Thus 
this statement is impossible to be false. It is a tautology.

\subsection{b}
We once again assume that this is not a tautology, then $p \implies (q \lor r)$ must be true and 
$(p \implies q) \lor (p \implies r)$ is false. Since this is a disjunction, then both 
arguments must be false, thus we have $p = 1, q = r = 0$. However, this valuation makes 
the former false. Therefore there is no way for this statement to be false, and it is a tautology.

\subsection{c}
Let $p = 0, q = 1$, we have $q \implies p \equiv 0$, $p \implies 0 \equiv 1$, $1 \implies p \equiv 0$. 
We have found a valuation such that it is false, and it is not a tautology.

\subsection{d}
Assume that this is not a tautology. Then $\neg p$ must be true and $p \implies q$ must be false. 
The former logically implies that $p$ is false. However this makes the latter always true. 
Since there is no way for this to be false, it is a tautology.


\section{Q6}
\subsection{a}
Assume that this is satisfiable, then each subpart of the conjunction must be true.
$p = 1$ from the first argument, $q = 1$ from the second argument. This valuation fails 
on the third argument. Since there is no way for the third argument to evaluate to true 
when the first two is true, this is not satisfiable.

\subsection{b}
Assume that this is satisfiable, then each subpart of the conjunction must be true.
$p = 1$ from the first argument. Since $\neg(q \implies (q \implies \neg p))$ must also 
be true, then $q \implies (q \implies \neg p)$ must be false. Therefore $q$ must be true and 
$(q \implies \neg p)$ false. Our valuation of $p = 1, q = 1$ satisfies this. Therefore when $p = 1. q =1$, 
the whole expression evaluates to true. It is satisfiable.

\subsection{c}
This expression is equivalent to $p \land \neg p$, which is not satisfiable.

\subsection{d}
Assume that this is satisfiable, then each subpart of the conjunction must be true.
From the second and third arguments we know that since the negative of the implication is 
true, then the implication must be false. $p = 1, r = 1, q = 0$. This also satisfies the 
first argument. When $p = 1, r = 1, q = 0$, this statement is true. It is satisfiable.


\section{Q7}
\subsection{a}
$$\neg(p \lor (q \implies r))$$
$$\equiv \neg(p \lor (r \lor \neg q))$$
$$\equiv \neg (p \lor r \lor \neg q)$$
$$\equiv \neg p \land \neg r \land q$$

\subsection{b}
$p \implies (q \land (p \implies r)) \equiv p \implies (q \land (\neg p \lor r)) \equiv \neg p \lor (q \land (\neg p \lor r))$
\newline
$\equiv \neg p \lor \neg \neg (q \land (\neg p \lor r)) = \neg p \lor \neg (\neg q \lor \neg (\neg p \lor r))$

\subsection{c}
$(p \land q) \iff (q \implies r) \equiv ((p \land q) \implies (q \implies r)) \land ((q \implies r) \implies (p \land q)) $
\newline
$$\equiv \neg \neg (((p \land q) \implies (q \implies r)) \land ((q \implies r) \implies (p \land q)))$$
$$\equiv \neg (\neg((p \land q) \implies (q \implies r)) \lor \neg((q \implies r) \implies (p \land q)))$$
$$\equiv \neg (((p \land q) \implies (q \implies r)) \implies \neg((q \implies r) \implies (p \land q)))$$
$$\equiv \neg ((\neg \neg(p \land q) \implies (q \implies r)) \implies \neg((q \implies r) \implies \neg \neg(p \land q)))$$
$$\equiv \neg ((\neg (\neg p \lor \neg q) \implies (q \implies r)) \implies \neg((q \implies r) \implies \neg (\neg p \lor \neg q)))$$
$$\equiv \neg ((\neg (p \implies \neg q) \implies (q \implies r)) \implies \neg((q \implies r) \implies \neg (p \implies \neg q)))$$


\section{Q8}
Assume that $S(\phi) \equiv \phi$ and $S(\psi) \equiv \psi$. Consider $S(\phi \land \psi)$, 
by definition it is equal to $\neg (\neg S(\phi) \lor \neg S(\psi) )$
\newline
By our inductive hypothesis this is now equivalent to $\neg (\neg \phi \lor \neg \psi )$, now we 
can simply apply De Morgan's Law to see that it is equivalent to $\phi \land \psi$


\section{Q9}
$\hat{V}(p) = \hat{V}(p)$
\newline
$\hat{V}(\neg p) = \hat{V}(p \downarrow p) $
\newline
$\hat{V}(p \lor q) = \hat{V}((p \downarrow q) \downarrow(p \downarrow q)) $
\newline
$\hat{V}(p \land q) = \hat{V}(((p \downarrow p) \downarrow(q \downarrow q)) 
\downarrow ((p \downarrow p) \downarrow(q \downarrow q))) $
\newline
$\hat{V}(p \implies q) = 
\hat{V}(((p \downarrow p)\downarrow q) \downarrow((p \downarrow p) \downarrow q)) $
\newline
$\hat{V}(p \iff q) = 
\hat{V}(((((p \downarrow p)\downarrow q) \downarrow((p \downarrow p) \downarrow q)\downarrow ((p \downarrow p)\downarrow q) \downarrow((p \downarrow p) \downarrow q))
 \downarrow(((q \downarrow q)\downarrow p) \downarrow((q \downarrow q) \downarrow p) \downarrow ((q \downarrow q)\downarrow p) \downarrow((q \downarrow q) \downarrow p))) 
\downarrow ((((p \downarrow p)\downarrow q) \downarrow((p \downarrow p) \downarrow q) \downarrow ((p \downarrow p)\downarrow q) \downarrow((p \downarrow p) \downarrow q))
 \downarrow(((q \downarrow q)\downarrow p) \downarrow((q \downarrow q) \downarrow p) \downarrow ((q \downarrow q)\downarrow p) \downarrow((q \downarrow q) \downarrow p))))$


\section{Q9}
We use a truth table:
\newline
\begin{tabular}{|c | c |c |c |c |}
    \hline
    $x_1$ & $x_2$ & $x_3$ & $\min^3(x_1, x_2, x_3)$ & $(x_1 \downarrow x_2) \lor (x_1 \downarrow x_3) \lor (x_2 \downarrow x_3)$ \\
    \hline
    0& 0& 0& 1& 1 \\
    \hline
    0& 0& 1& 1& 1 \\
    \hline
    0& 1& 0& 1& 1 \\
    \hline
    0& 1& 1& 0& 0 \\
    \hline
    1& 0& 0& 1& 1 \\
    \hline
    1& 0& 1& 0& 0 \\
    \hline
    1& 1& 0& 0& 0 \\
    \hline
    1& 1& 1& 0& 0 \\
    \hline
\end{tabular}
\newline
Thus $\min^3(x_1, x_2, x_3) \equiv (x_1 \downarrow x_2) \lor (x_1 \downarrow x_3) \lor (x_2 \downarrow x_3)$
\end{document}