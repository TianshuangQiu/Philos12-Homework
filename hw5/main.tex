\documentclass[12pt]{article}
\usepackage[usenames]{color} %used for font color
\usepackage{amsmath, amssymb, amsthm}
\usepackage{wasysym}
\usepackage[utf8]{inputenc} %useful to type directly diacritic characters
\usepackage{graphicx}
\usepackage{caption}
\usepackage{subcaption}
\usepackage{float}
\usepackage{mathtools}
\usepackage [english]{babel}
\usepackage [autostyle, english = american]{csquotes}
\MakeOuterQuote{"}
\graphicspath{ {./} }
\newcommand{\Z}{\mathbb{Z}}
\newcommand{\N}{\mathbb{N}}
\newcommand{\R}{\mathbb{R}}
\newcommand{\Q}{\mathbb{Q}}
\newcommand{\prob}{\mathbb{P}}
\newcommand{\degrees}{^{\circ}}
\DeclarePairedDelimiter\ceil{\lceil}{\rceil}
\DeclarePairedDelimiter\floor{\lfloor}{\rfloor}

\author{Tianshuang (Ethan) Qiu}
\begin{document}
\title{Philosophy 12, Problem Set 5}
\maketitle

\section{Q1}

\subsection{a}
Resolvent of $c1, c2$, we have $V_1 = q \lor s$
\newline
Resolvent of $c1, c3$, we have $V_2 = p \lor \neg s$
\newline
Resolvent of $c2, c3$, we have $V_3 = \neg p \lor \neg q$
\newline
We have our new formula: $(p \lor q) \land (\neg p \lor s) \land (\neg q \lor \neg s) 
\land (q \lor s) \land (p \lor \neg s) \land (\neg p \lor \neg q)$
\newline
We continue the algorithm: the resolvent of $c1, c4$ is $(p \lor q \lor s)$

\end{document}