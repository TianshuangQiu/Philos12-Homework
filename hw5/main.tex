\documentclass[12pt]{article}
\usepackage[usenames]{color} %used for font color
\usepackage{amsmath, amssymb, amsthm}
\usepackage{wasysym}
\usepackage[utf8]{inputenc} %useful to type directly diacritic characters
\usepackage{graphicx}
\usepackage{caption}
\usepackage{subcaption}
\usepackage{float}
\usepackage{mathtools}
\usepackage [english]{babel}
\usepackage [autostyle, english = american]{csquotes}
\MakeOuterQuote{"}
\graphicspath{ {./} }
\newcommand{\Z}{\mathbb{Z}}
\newcommand{\N}{\mathbb{N}}
\newcommand{\R}{\mathbb{R}}
\newcommand{\Q}{\mathbb{Q}}
\newcommand{\prob}{\mathbb{P}}
\newcommand{\degrees}{^{\circ}}
\DeclarePairedDelimiter\ceil{\lceil}{\rceil}
\DeclarePairedDelimiter\floor{\lfloor}{\rfloor}

\author{Tianshuang (Ethan) Qiu}
\begin{document}
\title{Philosophy 12, Problem Set 5}
\maketitle

\section{Q1}

\subsection{a}
Resolvent of $c1, c2$, we have $V_1 = q \lor s$
\newline
Resolvent of $c1, c3$, we have $V_2 = p \lor \neg s$
\newline
Resolvent of $c2, c3$, we have $V_3 = \neg p \lor \neg q$
\newline
We have our new formula: $(p \lor q) \land (\neg p \lor s) \land (\neg q \lor \neg s) 
\land (q \lor s) \land (p \lor \neg s) \land (\neg p \lor \neg q)$
\newline
We continue the algorithm: the resolvent of $c1, c4$ is $(p \lor q \lor s)$, $c1, c5$
yields $(p \lor q \lor \neg s)$, $c1, c6$ yields $(q \lor \neg q) = 1$, and will be excluded.
\newline
$c2, c4$ yields $(\neg p \lor q \lor s)$, $c2, c5$ yields $1$, $c2, c6$ yields $(s \lor \neg p \lor \neg q)$
\newline
$c3, c4$ yields $1$, $c3, c5$ yields $(\neg s \lor p \lor \neg q)$, $c3, c6$ yields $(\neg s \lor \neg q \lor \neg p)$
\newline
Our new formula becomes $(p \lor q) \land (\neg p \lor s) \land (\neg q \lor \neg s) 
\land (q \lor s) \land (p \lor \neg s) \land (\neg p \lor \neg q) 
\land (p \lor q \lor s) \land (p \lor q \lor \neg s) \land (\neg p \lor q \lor s)
\land (s \lor \neg p \lor \neg q) \land (\neg p \lor q \lor s) \land (s \lor \neg p \lor \neg q)
\land (\neg s \lor p \lor \neg q) \land (\neg s \lor \neg q \lor \neg p)$
\newline
At this stage, any new resolution of any two sub-formulas result in an expression already 
in our formula. Thus our algorithm terminates, and the formula is satisfiable.

\subsection{b}
Resolvent of $c1, c2$: $(q)$
\newline
Resolvent of $c1, c3$: $(\neg r)$
\newline
Resolvent of $c1, c4$: $(r \lor \neg q)$
\newline
Consider our formula: $p \land (\neg p \lor q) \land (\neg p \lor \neg r) \land (r \lor \neg p \lor \neg q)
\land (q) \land (\neg r) \land (r \lor \neg q)$
\newline
Resolvent of $q \land (r \lor \neg q)$: $r$. We have both $\neg r$ and $r$ in our CNF, 
thus it is not satisfiable.


\section{Q2}
\subsection{a}
\[
    (p \lor \neg s) \land (\neg p \lor q \lor s) \land \neg s \land (s \lor \neg q)
    \]
Resolvent of $c1, c3$: $\neg s$ ($c1$ is subsumed)
\newline
Resolvent of $c2, c4$: $(\neg p \lor s)$ ($c2$ is subsumed)
\newline
Our formula becomes: $\neg s \land (s \lor \neg q) \land (\neg p \lor s)$
\newline
Resolvent of $c1, c2$: $\neg q$ ($c2$ is subsumed)
\newline
Resolvent of $c1, c3$: $\neg p$
\newline
Thus our formula is $\neg s \land \neg q \land \neg p$. There can no longer be more 
resolvents. Our algorithm terminates and thus the expression is satisfiable.

\subsection{b}
$s = 0, q = 0, p = 0$


\section{Q3}
Assume that the algorithm fails and that the result after subsumption yields that the expression 
is satisfiable but in reality it is not.
\newline
Then there must exist subformulas $C, C'$ such that $C \leq C'$. Our algorithm removes $C'$ when 
$C$ is in the formula. Our assumption implies that at least one such step fails.
\newline
Consider an arbitrary valuation of $C$ such that $\hat{V}(C) = 1$. Since $C'$ is 
subsumed by $C$, it is a disjunction of several other atomic expressiosn. Thus we have 
$C' = C \lor ...$. Now we can subsitute the valuation of $C$ that gives $\hat{V}(C) = 1$. 
From our formula above, we have $C' = 1 \lor ... = 1$.
\newline
Thus it is impossible for any subsuming formula to be satisfiable and the original to be not 
satisfiable. Our assumption is in correct and the algorithm works.
\end{document}