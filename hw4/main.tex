\documentclass[12pt]{article}
\usepackage[usenames]{color} %used for font color
\usepackage{amsmath, amssymb, amsthm}
\usepackage{wasysym}
\usepackage[utf8]{inputenc} %useful to type directly diacritic characters
\usepackage{graphicx}
\usepackage{caption}
\usepackage{subcaption}
\usepackage{float}
\usepackage{mathtools}
\usepackage [english]{babel}
\usepackage [autostyle, english = american]{csquotes}
\MakeOuterQuote{"}
\graphicspath{ {./} }
\newcommand{\Z}{\mathbb{Z}}
\newcommand{\N}{\mathbb{N}}
\newcommand{\R}{\mathbb{R}}
\newcommand{\Q}{\mathbb{Q}}
\newcommand{\prob}{\mathbb{P}}
\newcommand{\degrees}{^{\circ}}
\DeclarePairedDelimiter\ceil{\lceil}{\rceil}
\DeclarePairedDelimiter\floor{\lfloor}{\rfloor}

\author{Tianshuang (Ethan) Qiu}
\begin{document}
\title{Philosophy 12, Problem Set 5}
\maketitle

\section{Q1}
The function is true if and only if there are between (inclusive) 2 and 3 true values in $x_1, x_2, ... ,x_5$.
Therefore we can simply consider the following cases: there are 2 trues or (exclusive) there are 3 trues.
\newline
For an algorithmic approach, we can simply negate the cases where they are negative, or more 
explicity, when they sum to 0, 1, 4 or 5.
\newline
$f(x_1, x_2, x_3,x_4,x_5) = 0$, then we have $\neg (x_1 \lor x_2 \lor x_3 \lor x_4 \lor x_5)$
\newline
$f(x_1, x_2, x_3,x_4,x_5) = 1$, then we have
\begin{itemize}
    \item $\neg (x_2 \lor x_3 \lor x_4 \lor x_5) \land x_1$
    \item $\neg (x_1 \lor x_3 \lor x_4 \lor x_5) \land x_2$
    \item $\neg (x_1 \lor x_2 \lor x_4 \lor x_5) \land x_3$
    \item $\neg (x_1 \lor x_2 \lor x_3 \lor x_5) \land x_4$
    \item $\neg (x_1 \lor x_2 \lor x_3 \lor x_4) \land x_5$
\end{itemize}

$f(x_1, x_2, x_3,x_4,x_5) = 4$, then we have
\begin{itemize}
    \item $ x_2 \land x_3 \land x_4 \land x_5 \land  x_1$
    \item $ x_1 \land x_3 \land x_4 \land x_5 \land  x_2$
    \item $ x_1 \land x_2 \land x_4 \land x_5 \land  x_3$
    \item $ x_1 \land x_2 \land x_3 \land x_5 \land  x_4$
    \item $ x_1 \land x_2 \land x_3 \land x_4 \land  x_5$
\end{itemize}

$f(x_1, x_2, x_3,x_4,x_5) = 5$, then we have $x_1 \land x_2 \land x_3 \land x_4 \land x_5$
\newline
Finally we combine them and negate the whole thing:
$$\neg((\neg (x_1 \lor x_2 \lor x_3 \lor x_4 \lor x_5)) \lor (\neg (x_2 \lor x_3 \lor x_4 \lor x_5) \land x_1)
\lor (\neg (x_1 \lor x_3 \lor x_4 \lor x_5) \land x_2)$$$$ \lor (\neg (x_1 \lor x_2 \lor x_4 \lor x_5) \land x_3)
\lor (\neg (x_1 \lor x_2 \lor x_3 \lor x_5) \land x_4) \lor (\neg (x_1 \lor x_2 \lor x_3 \lor x_4) \land x_5)
\lor (x_2 \land x_3 \land x_4 \land x_5 \land  x_1) $$$$\lor (\neg (x_1 \lor x_3 \lor x_4 \lor x_5) \land x_2)
\lor (x_1 \land x_2 \land x_4 \land x_5 \land  x_3) \lor (x_1 \land x_2 \land x_3 \land x_5 \land  x_4)
\lor (x_1 \land x_2 \land x_3 \land x_4 \land  x_5) $$$$\lor (x_1 \land x_2 \land x_3 \land x_4 \land x_5))$$


\section{Q5}

\subsection{a}
$\neg((p\land q\land r)\lor (\neg p\land q\land r)\lor (\neg q\land \neg r))$
\newline
$\equiv \neg(p\land q\land r)\land \neg (\neg p\land q\land r)\land \neg(\neg q\land \neg r)$
\newline
$\equiv (\neg p \lor q \lor r) \land (p \lor q \lor r) \land (q \lor r)$

\subsection{b}
$(p\to (q\land r))\land \neg (q\leftrightarrow r)\land ((q\lor r)\to p)$
\newline
$\equiv (\neg p \lor (q \land r)) \land \neg((\neg q \lor r) \land (\neg r \lor q)) \land (\neg (q \lor r) \lor p)$
\newline
$\equiv (\neg p \lor q) \land (\neg p \lor r) \land (\neg(\neg q \lor r) 
\lor \neg(\neg r \lor q)) \land (p \lor (\neg q \land \neg r))$
\newline
$\equiv (\neg p \lor q) \land (\neg p \lor r) \land ((\neg r \land q) \lor (r \land \neg q)) \land (p \lor \neg q) \land (p \lor \neg r) $
\newline
$((\neg r \land q) \lor (r \land \neg q))$
$\equiv ((\neg r \land q) \lor r) \land ((\neg r \land q) \lor \neg q)$
\newline
$\equiv (\neg r \lor r) \land (r \lor q) \land (\neg q \lor \neg r) \land (\neg q \lor q)$
\newline
Therefore the final expression is
$$(\neg p \lor q) \land (\neg p \lor r) \land  (r \lor q) \land (\neg q \lor \neg r) \land (p \lor \neg q) \land (p \lor \neg r) $$

\subsection{c}
$(p \implies (q \implies r)) \implies ((p \implies q) \implies (p \implies r))$
\newline
$\equiv \neg(\neg p \lor (\neg q \lor r)) \lor (\neg(\neg p \lor q) \lor (\neg p \lor r))$
\newline
$\equiv (p \land \neg(\neg q \lor r)) \lor ( p \land \neg q) \lor (\neg p \lor r)$
\newline
$\equiv (p \land q \land \neg r) \lor ( p \land \neg q) \lor \neg p \lor r$
\newline
$\equiv ((p) \land (p \lor \neg q) \land (q \lor p) \land (\neg r \lor p) \land (\neg r \lor \neg q)) \lor \neg p \lor r$
\newline
$\equiv ((p) \land (p \lor \neg q) \land (\neg r \lor \neg q \lor \neg p)) \lor r$
\newline
$\equiv (p \lor r) \land (p \lor \neg q \lor r)$

A lot of terms are removed since $1 \land x \equiv x$, and $p \lor \neg p \equiv 1$.

\section{Q6}
\subsection{a}
Write the complete truth table of the formula, 
then for each row that the formula is true, 
write clause that is the conjunction of all the 
truth values of the variables. Finally combine each 
row clause with a disjunction. Thus we have a 
DNF that describes the formula and is equivalent to it.


\subsection{b}
$(p \iff q) \iff r$
\newline
$\equiv \neg(\neg p \lor q) \lor r$
\newline
$\equiv (p \land \neg q) \lor r$
\newline
$(p \lor q) \land (p \lor r) \land (p \lor s)$
\newline
$\equiv (p \lor (q \land r)) \land (p \lor s)$
\newline
$\equiv p \lor (q \land r \land s)$
\end{document}