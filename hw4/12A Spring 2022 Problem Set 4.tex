\documentclass[english]{article}
\usepackage[T1]{fontenc}
\usepackage[latin9]{inputenc}
\usepackage{amsmath,amsthm} 
\usepackage{graphicx}
\usepackage{amssymb}
\usepackage{enumitem}
\usepackage{natbib} 
\usepackage{bussproofs}
\usepackage[letterpaper]{geometry} 
\geometry{verbose,tmargin=1in,bmargin=1in,lmargin=1.5in,rmargin=1.5in} 
\usepackage{appendix}
\usepackage{tikz}
\usepackage{pgf} 
\usetikzlibrary{patterns,automata,arrows,shapes,snakes,topaths,trees,backgrounds,positioning,through,calc}
\usepackage{setspace}
\usepackage{multicol}
\usepackage{graphicx}
\usepackage{stmaryrd}  
\usepackage{comment}
\usepackage{hyperref}
\usepackage{circuitikz}
%\hypersetup{colorlinks=true, citecolor=false, linkcolor=blue}  

\theoremstyle{definition}
\newtheorem{theorem}{Theorem}
\newtheorem{fact}{Fact}
\newtheorem{proposition}{Proposition}
\newtheorem{example}{Example}
\newtheorem{definition}{Definition}
\newtheorem{lemma}{Lemma}
\newtheorem{question}{Question}  
\newtheorem{remark}{Remark} 

\usepackage{enumitem}


\makeatletter  
\makeatother

\onehalfspace

\begin{document} 
	
	\title{PHIL 12A -- Spring 2022 \\ Problem Set 4}
	\date{Due February 20, 2022}
	
	\maketitle
	
	%\tableofcontents
	
	\begin{center}75 points.\end{center}
	
	\setcounter{section}{1}
	
	\section{Basic Theory of Propositional Logic}
	
	\subsection{Truth Functions}
	
	\subsubsection{Truth Functions I}
	\begin{enumerate}[label=\arabic*.,ref=\arabic*,resume]
		\item (15 points) Consider the truth function $\mathrm{full}:2^5\to 2$ (short for `full house'):
		\[\mathrm{full}(x_1,x_2,x_3,x_4,x_5)=\begin{cases} 1&\mbox{ if }2\leq x_1+x_2+x_3+x_4+x_5\leq 3 \\ 0 &\mbox{ otherwise}   \end{cases}.\]
		Find a formula that defines $\mathrm{full}$.
	\end{enumerate}
	\subsubsection{Truth Functions II}
	
	\begin{enumerate}[label=\arabic*.,ref=\arabic*,resume]
		\item \textit{Extra credit}.  (5 points) Let $\mathcal{L}_\to(\{p,q\})$ be the set of formulas of $\mathcal{L}(\{p,q\})$ in which the only connective is $\to$. How many \textit{equivalence classes} of formulas of $\mathcal{L}_\to(\{p,q\})$ are there? Justify your answer.
	\end{enumerate}
	
	\subsubsection{Digital Circuits}
	\begin{enumerate}[label=\arabic*.,ref=\arabic*,resume]
		\item (10 points) Draw circuits corresponding to the following formulas that use only NOT, AND, and OR gates:
		\begin{enumerate}
			\item $p\to(\neg q\to (p\land r))$;
			\item $\neg((p\lor r)\to (q\lor r))$
		\end{enumerate}
		\newpage	
		\item\label{CircProb2} (10 points) Given the circuit in Figure \ref{Circ2}, draw an equivalent circuit using only NOR gates. 
	\end{enumerate}
		
	\begin{figure}[h]
		\begin{center}
			\begin{circuitikz} \draw
				(0,2) node[and port] (myand1) {}
				(-.7,2) node {{\tiny AND}}
				(-.8,0) node {{\tiny NOT}}
				(1.45,1) node {{\tiny OR}}
				(-.65,0) node[not port] (myand2) {}
				(2,1) node[or port] (myxnor) {}
				(myand1.out) -- (myxnor.in 1)
				(myand2.out) -- (myxnor.in 2);
			\end{circuitikz}
			\caption{circuit for problem \ref{CircProb2}.}\label{Circ2}
		\end{center}
	\end{figure}
	
	
	\subsection{Algorithms and Combinatorial Problems}
	\subsubsection{Algorithms I}
	\begin{enumerate}[label=\arabic*.,ref=\arabic*,resume]
		\item\label{CNFprob} (15 points) Convert each of the following formulas into an equivalent formula in CNF, showing each step of the CNF algorithm:
		\begin{enumerate}
			\item $\neg((p\land q\land r)\lor (\neg p\land q\land r)\lor (\neg q\land \neg r))$;
			\item $(p\to (q\land r))\land \neg (q\leftrightarrow r)\land ((q\lor r)\to p)$;
			\item $(p\to(q\to r))\to((p\to q)\to(p\to r))$.
		\end{enumerate}
		\item (15 points)
		\begin{enumerate}
			\item Describe an algorithm for converting any given formula into an equivalent formula in DNF. 
			\item Run your algorithm on the following formulas:
			\begin{enumerate}
				\item $(p\leftrightarrow q)\leftrightarrow r$;
				\item $(p\lor q)\land (p\lor r)\land (p\lor s)$.
			\end{enumerate}
		\end{enumerate}
	\end{enumerate}
	
	\subsubsection{Algorithms II}
	
	\begin{enumerate}[label=\arabic*.,ref=\arabic*,resume]
		\item (10 points) On the resolution algorithm. For each of the CNF formulas produced in problem \ref{CNFprob}, use resolution  to test whether the formulas is satisfiable, showing each step of the resolution algorithm.
		%\item (\emph{Practice}) On the tableau algorithm.  Exercises 8.1, 8.2, 8.3, and 8.4 of Logic in Action.
	\end{enumerate}
	
\end{document}



