\documentclass[12pt]{article}
\usepackage[usenames]{color} %used for font color
\usepackage{amsmath, amssymb, amsthm}
\usepackage{wasysym}
\usepackage[utf8]{inputenc} %useful to type directly diacritic characters
\usepackage{graphicx}
\usepackage{caption}
\usepackage{subcaption}
\usepackage{float}
\usepackage{mathtools}
\usepackage [english]{babel}
\usepackage [autostyle, english = american]{csquotes}
\MakeOuterQuote{"}
\graphicspath{ {./} }
\newcommand{\Z}{\mathbb{Z}}
\newcommand{\N}{\mathbb{N}}
\newcommand{\R}{\mathbb{R}}
\newcommand{\Q}{\mathbb{Q}}
\newcommand{\prob}{\mathbb{P}}
\newcommand{\degrees}{^{\circ}}
\DeclarePairedDelimiter\ceil{\lceil}{\rceil}
\DeclarePairedDelimiter\floor{\lfloor}{\rfloor}

\author{Tianshuang (Ethan) Qiu}
\begin{document}
\title{Philosophy 12, Problem Set 6}
\maketitle

\section{Q1}
\subsection{a}
Randomly pick items into the bag until we overshoot the maximumt weight $W$. Then
remove the last item and check if what's in hte bag has at least $V$ calories. If
it does, we have shown that $S$ exists. If not we take every item out and repeat
the algorithm.
\newline
This is not a polynomial time algorithm. Since we are randomly selecting the items,
it is possible to never find such a set $S$ even if $S$ exists.

\subsection{b}
Since each item can either be picked or not picked, there are $2^n$ ways to pick
our items for the backpack. Now we can simply iterate through all $2^n$ options to
see if any fits the requirement of having less weight than $W$ and more calories than
$V$.
\newline
This is also not a polynomial time algorithm since $2^n$ is exponential.


\section{Q2}
\subsection{Each country has at least one color}
Let $i$ be an arbitrary country, then we have $c_i \lor m_i \lor y_i$. Therefore we can apply
to all $i$:
\[\bigwedge _{i = 1} ^n (c_i \lor m_i \lor y_i)\]

\subsection{Each country has at most one color}
Let $i$ be an arbitrary country, then we have $(c_i \land \neg m_i \land \neg y_i) \lor
(m_i \land \neg y_i \land \neg c_i) \lor (y_i \land \neg c_i \land \neg m_i)$.
Therefore we can apply
to all $i$:
\[\bigwedge _{i = 1} ^n (c_i \land \neg m_i \land \neg y_i) \lor
(m_i \land \neg y_i \land \neg c_i) \lor (y_i \land \neg c_i \land \neg m_i)\]

\subsection{No adjacent countries have the same color}
If two adjacent countries have the same color then $(c_i \land c_j) \lor (m_i \land m_j) \lor (y_i \land y_j)$
We can negate that for all such $i,j$
\[\neg\bigvee_{i,j \text{ adjacent}}(c_i \land c_j) \lor (m_i \land m_j) \lor (y_i \land y_j)\]
\[\equiv \bigwedge_{i,j \text{ adjacent}}\neg(c_i \land c_j) \land \neg(m_i \land m_j) \land \neg (y_i \land y_j)\]

Finally we can simply combine these, so our answer is $(2.1)\land(2.2)\land(2.3)$

\section{Q5}
The first statement is true because there are only two Republicans in the race:
Reagan and Anderson. The second statement is where the problem begins. "A Republican"
will win the election is true in the pollee's mind because Regan is a Republican.
They have subsituted "a republican" with "Reagan" because he is decisively
ahead of Carter. Therefore the final conclusion: "If it's not Regan who wins, it will
be Anderson" doesn't make sense.
\newline
This shows that everyday English is not as strict as formal logic and contains a lot
of implicit subsitutions of terms.
\end{document}
